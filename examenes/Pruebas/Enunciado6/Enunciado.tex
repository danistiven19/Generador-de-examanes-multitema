\documentclass[a4paper,12pt]{article}\usepackage[spanish]{babel}\usepackage{times}

\title{Vista preliminar de Enunciado}
\usepackage{graphicx}
\begin{document}\twocolumn 

\maketitle

Texto Uno \newline
     1. Gracias a la literatura, a las conciencias que formó, a los deseos y anhelos que inspiró, al desencanto de lo real con que volvemos del viaje a una bella fantasía, la civilización es ahora menos cruel que cuando los contadores de cuentos comenzaron a humanizar la vida con sus fábulas. Seríamos peores de lo que somos sin los buenos libros que leímos, más conformistas, menos inquietos e insumisos y el espíritu crítico, motor del progreso, ni siquiera existiría. Igual que escribir, leer es protestar contra las insuficiencias de la vida. Quien busca en la ficción lo que no tiene, dice, sin necesidad de decirlo, ni siquiera saberlo, que la vida tal como es no nos basta para colmar nuestra sed de absoluto, fundamento de la condición humana, y que debería ser mejor. Inventamos las ficciones para poder vivir de alguna manera las muchas vidas que quisiéramos tener cuando apenas disponemos de una sola.\newline
\end{document}
