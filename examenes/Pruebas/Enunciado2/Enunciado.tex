\documentclass[a4paper,12pt]{article}\usepackage[utf8]{inputenc}\usepackage[spanish]{babel}\usepackage{times}

\title{Vista preliminar de Enunciado}
\usepackage{graphicx}
\begin{document}\twocolumn 

\maketitle

\textbf{ Texto Dos } \newline 1. Un escritor, sin duda, es un especulador. Alguien que insatisfecho con la realidad se aferra a pequeños momentos de la vida o la Historia y los encamina con palabras por \underline{ vericuetos } que también pudieron ser. La literatura es una eterna suposición, es una coartada contra el destino, el cual, así se haya vivido, no indica que todo tenga que ser como fue. La literatura es \underline{ tal vez } uno de los pocos caminos donde la imaginación tiene vía libre. \newline 2. \underline{ Es por eso} que la suposición de cosas ha hecho que en libros como \textit{ La conjura contra América } Philip Roth imagine cómo serían los Estados Unidos si en vez del presidente Roosevelt hubiera sido elegido el antisemita Lindbergh, quien al hacer un pacto de no agresión con Hitler se dedica a perseguir judíos en el país americano. O que Don Delillo explore en su libro \textit{ Fascinación} qué pasaría si fuera cierto que el mismo Hitler protagonizó una película pornográfica que fue filmada durante sus últimos días dentro del búnker en Berlín, cuando el Ejército Rojo se acercaba y la ciudad era bombardeada. \newline  3. A propósito del premio Alfaguara de Novela, que bien merecido lo ganó este año Juan Gabriel Vásquez, recuerdo que en la presentación de su novela  \textit{ Historia secreta de la Costaguana } en el 2007, Vásquez dijo que la idea le surgió cuando estaba escribiendo una pequeña biografía sobre Joseph Conrad y se dio cuenta de que posiblemente el escritor polaco había leído el libro de Pérez Triana, \textit{De Bogotá al Atlántico}, que al parecer le sirvió para escribir  \textit{ Nostromo }; desde entonces Juan Gabriel empezó a suponer una serie de situaciones adicionales para su novela que involucraron un período de la historia de Colombia, la construcción del canal de panamá y, desde luego, una parte de la vida de Conrad. La Historia, con mayúscula, para Juan Gabriel se volvió una historia con minúscula. \newline 4. Ricardo Pliglia en su libro de ensayos \textit{ El último lector} dice que un lector es también el que lee mal, distorsiona, percibe confusamente. “En la clínica del arte de leer no siempre el que tiene mejor vista lee mejor”, agrega el escritor argentino. De alguna forma esta distorsión también la podríamos aplicar al novelista, quien, a diferencia del historiador, no lo atan las fechas exactas, las glorias, ni mucho menos los nombres de los ilustres protagonistas con todas sus cualidades y virtudes. Al contrario, en la literatura los hilitos de las costuras históricas cuelgan a la espera de que los escritores las halen para especular, para suponer, para hacer más rica y emocionante la vida misma. Balzac decía que la novela era la historia privada de las naciones. \newline 5. Cuando se lee literatura lo mejor es no acercarse con un diccionario enciclopédico para señalar al margen la supuesta ignorancia del novelista que modifica un dato o le pone una nariz que no era a un general cualquiera; cuando se lee literatura es porque la mente está abierta a observar la Historia y la historia con los ojos del asombro, así con el tiempo se crea más en la existencia del David Copperfield de Charles Dickens, que en el mismo mago de Nueva Jersey quien, tal vez, no es más que una ilusión. \newline \textsl{ ARISTIZÁBAL, Diego. La literatura como especulación. El Colombiano. Medellín, 24 de marzo de 2011, p. 4a } \newline
\end{document}
